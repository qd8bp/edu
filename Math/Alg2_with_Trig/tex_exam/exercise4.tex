
\begin{question}
A thief is filling her backpack with two types of valuable substances.
She can carry up to 15 kg, and her backpack can fit up to 10 liters.

Each bag of \(X\) has a weight of 0.6 kg, volume of 0.6 L, and value of
100 thousand USD.

Each bag of \(Y\) has a weight of 0.2 kg, volume of 0.1 L, and value of
4 thousand USD.

There is no requirement to take full bags, so the thief can opt for a
fraction of a bag.

How many bags of each should the thief take to maximize her profit?
\end{question}

\begin{solution}
We write a weight inequality. \[0.6x+0.2y \le 15\] We write a volume
inequality. \[0.6x+0.1y \le 10\] We graph the two inequalities, shading
the feasible region.

\includegraphics{unnamed-chunk-1-1-4.pdf}\\

There are three vertices of interest. \[(0,75) \] \[(8.33,50) \]
\[(16.67,0)\]

We write a profit function (the objective function).
\[P(x,y) = 100x+4y \]

We determine the profits. \[P(0,75)=300 \] \[P(8.33,50)=1033.33 \]
\[P(16.67,0)=1666.67 \]

Thus the thief should take 16.67 bags of \(X\) and 0 bags of \(Y\).
\end{solution}

